% https://www.rpi.edu/dept/arc/training/latex/resumes/

\documentclass[margin,line]{res}

\usepackage[utf8x]{inputenc} 
\usepackage{paralist}
\usepackage{hyperref}

\oddsidemargin -.5in
\evensidemargin -.5in
\textwidth=6.0in
\itemsep=0in
\parsep=0in

\newenvironment{list1}{
  \begin{list}{\ding{113}}{%
      \setlength{\itemsep}{0in}
      \setlength{\parsep}{0in} \setlength{\parskip}{0in}
      \setlength{\topsep}{0in} \setlength{\partopsep}{0in} 
      \setlength{\leftmargin}{0.17in}}}{\end{list}}
\newenvironment{list2}{
  \begin{list}{$\bullet$}{%
      \setlength{\itemsep}{0in}
      \setlength{\parsep}{0in} \setlength{\parskip}{0in}
      \setlength{\topsep}{0in} \setlength{\partopsep}{0in} 
      \setlength{\leftmargin}{0.2in}}}{\end{list}}


\begin{document}

\name{Frédéric Portaria-Janicki, B. Ing.\vspace*{.1in}}

\begin{resume}
  \section{\sc Information}
  \vspace{.05in}
  \begin{tabular}{@{}p{2in}p{4in}}
    {\it Mobile:}  (514) 969-9816      \\
    {\it Courriel:} fjanicki@gmail.com \\
  \end{tabular}


  \section{\sc Formation Académique}
   {\bf Baccalauréat en Génie Électrique et Informatique} \hfill {\bf 2015}\\
  École de technologie supérieure (ÉTS)
  Université du Québec, Montréal

  {\bf Diplôme d'études collégiales en sciences de la nature} \hfill {\bf 2010}\\
  Collège Jean-de-Brébeuf

  \section{\sc Expérience professionnelle}
   {\bf PVP Techonologies} \hfill {\bf Depuis 2017}\\
  Développeur R\&D - Mobile et systèmes embarqués\\
  \begin{itemize}
    \item Développement d'une solution de télématique dans le domaine de la mobilité partagée. La solution est développée en microservices pour un système embarqué Linux sous la plateforme Legato, en C++14. Le produit final est installé dans des voitures.
    \item Désign et développement d'une solution sans-fil BLE afin de commander un véhicule hors-ligne sur la plateforme Openfleet.
    \item Développement mobile Android moderne d'applications mobiles natives. Développement de la plateforme mobile Openfleet. Intégration d'une {\emph Clean Architecture} dans l'application mobile afin de faire une application en marque blanche.
    \item Organisation de formations technique en C++, Java et Kotlin à l'interne.
  \end{itemize}

  {\emph Technologies utilisées:} C++14, MQTT, Python, Kotlin, Java, TDD, Dagger, RxJava, RxCpp, Legato, Embedded Linux (Yocto), Boost C++, Docker

    {\bf Samsao} \hfill {\bf 2016}\\
  Développeur Mobile Android - Genie Conseil\\\\
  Développement mobile Android moderne d'applications mobiles natives. Developpement d'un systeme de flashing de firmware sans-fil via \emph{Bluetooth Low Energy} pour l'industrie automobile.
    {\emph Exemples de clients:} Directed Technologies.

    {\emph Technologies utilisées:} Java, Kotlin, TDD, Android Studio, Dagger, RxJava.

    {\bf nventive} \hfill {\bf 2015}\\
  Développeur Mobile - Ingénierie de développement logiciel\\\\
  Développement multiplateformes (Android/iOS/Windows Phone) d'applications mobiles pour clients sous Xamarin.
    {\emph Exemples de clients:} Rdio, Twitter.

    {\emph Technologies utilisées:} C\#, Xamarin, .NET 5, Android SDK, iOS, Git, Visual Studio Online.

    {\bf Mobeewave} \hfill {\bf 2015}\\
  Stagiaire - Ingénierie de développement logiciel\\\\
  Participation au développement d'une application de type serveur back-end en ajoutant de nouvelles fonctionnalitées et en participant au processus de design avec une méthodologie {\emph Agile}.
  {\emph Accomplissements:} Design et implémentation d'un système de cache partagée.

  {\emph Technologies utilisées:} C\#, .NET 4.5, Amazon AWS, Redis, WebApi, Git, TFS, Visual Studio Online.

    {\bf iBwave} \hfill {\bf 2014}\\
  Stagiaire - Ingénierie de développement logiciel\\\\
  Participation au développement de l'application {\emph iBwave design} en ajoutant de nouvelles fonctionnalitées et en participant au processus de design.

    {\emph Technologies utilisées:} C\#, XML, .NET 4.5

  %%%%%%%%%%%% AUTRES EXP %%%%%%%%%%%%%%

  \section{\sc Expérience\\ Divers}
   {\bf Projets}\hfill
  \begin{itemize}
    \item Design et construction d'un dôme géodesique dans le contexte d'un projet multidisplinaire artistique pour le festival OpenMind 2015. Projet récipiendaire d'une bourse pour son design écologique.
    \item Design et fabrication d'un mur de 500 LED avec contrôle interactif numérique. Présenté lors d'une soirée de Noël chez nventive ainsi qu'une soirée résautage organisée par Facebook.
          \\\\{\emph Technologies utilisées:} TouchDesigner, ARM Cortex-M4, C, Communication série, Découpage laser, Design electrique.
    \item Développement d'une plateforme me permettant de faire de la génération d'images 2D et 3D en temps-réel pour concerts de musique. (VJing).
  \end{itemize}
  {\bf Implication} \hfill
  \begin{itemize}
    \item Co-fondateur et directeur technique du collectif d'artistes {\emph Eden Creative}.
    \item Membre du club de développement de jeux vidéo {\emph Conjure}.
    \item Développement d'un prototype de jeu multijoueur dans le cadre du concours Ubisoft Académia 2014. Prototype gagnant de 2 des 10 prix disponibles.
    \item Participation à la compétition de sécurité informatique NorthSec 2014.
    \item Bénévole lors des jeux de Génie 2012.
  \end{itemize}
  %%%%%%%%%%% Mentions%%%%%%%%%%%%%%%%%

  %%%%%%%%%%% LOISIRS %%%%%%%%%%%%%%%%%

  \section{\sc Loisirs}
  Escalade - Yoga - VJing -  Informatique - Art visuel procédural - Photographie - Vélo

  %%%%%%%%%%%% FIN %%%%%%%%%%%%%%%%

\end{resume}
\end{document}




